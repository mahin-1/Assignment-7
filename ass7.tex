%%%%%%%%%%%%%%%%%%%%%%%%%%%%%%%%%%%%%%%%%%%%%%%%%%%%%%%%%%%%%%%
%
% Welcome to Overleaf --- just edit your LaTeX on the left,
% and we'll compile it for you on the right. If you open the
% 'Share' menu, you can invite other users to edit at the same
% time. See www.overleaf.com/learn for more info. Enjoy!
%
%%%%%%%%%%%%%%%%%%%%%%%%%%%%%%%%%%%%%%%%%%%%%%%%%%%%%%%%%%%%%%%

% Inbuilt themes in beamer
\documentclass{beamer}

%packages:
% \usepackage{tfrupee}
% \usepackage{amsmath}
% \usepackage{amssymb}
% \usepackage{gensymb}
% \usepackage{txfonts}

% \def\inputGnumericTable{}

% \usepackage[latin1]{inputenc}                                 
% \usepackage{color}                                            
% \usepackage{array}                                            
% \usepackage{longtable}                                        
% \usepackage{calc}                                             
% \usepackage{multirow}                                         
% \usepackage{hhline}                                           
% \usepackage{ifthen}
% \usepackage{caption} 
% \captionsetup[table]{skip=3pt}  
% \providecommand{\pr}[1]{\ensuremath{\Pr\left(#1\right)}}
% \providecommand{\cbrak}[1]{\ensuremath{\left\{#1\right\}}}
% %\renewcommand{\thefigure}{\arabic{table}}
% \renewcommand{\thetable}{\arabic{table}}      

\setbeamertemplate{caption}[numbered]{}

\usepackage{enumitem}
\usepackage{tfrupee}
\usepackage{amsmath}
\usepackage{amssymb}
\usepackage{gensymb}
\usepackage{graphicx}
\usepackage{txfonts}

\def\inputGnumericTable{}

\usepackage[latin1]{inputenc}                                 
\usepackage{color}                                            
\usepackage{array}                                            
\usepackage{longtable}                                        
\usepackage{calc}                                             
\usepackage{multirow}                                         
\usepackage{hhline}                                           
\usepackage{ifthen}
\usepackage{caption} 
\captionsetup[table]{skip=3pt}  
\providecommand{\pr}[1]{\ensuremath{\Pr\left(#1\right)}}
\providecommand{\cbrak}[1]{\ensuremath{\left\{#1\right\}}}
\renewcommand{\thefigure}{\arabic{table}}
\renewcommand{\thetable}{\arabic{table}}   
\providecommand{\brak}[1]{\ensuremath{\left(#1\right)}}
\providecommand{\brak}[1]{\ensuremath{\left(#1\right)}}

% Theme choice:
\usetheme{CambridgeUS}

% Title page details: 
\title{Assignment 7} 
\author{MAHIN BANSAL \\CS21BTECH11034}
\date{20, July 2022}
\logo{\large \LaTeX{}}

\begin{document}

% Title page frame
\begin{frame}
    \titlepage 
\end{frame}

% Remove logo from the next slides
\logo{}




% Lists frame
\section{Question}
\begin{frame}{Question}
    The process x(t) is W.S.S. (\textit{Wide Sense Stationary}) with $R_{XX}(t) = 5 \delta (t)$ and 
    \begin{align}
        y`(t) + 2y(t) = x(t)
    \end{align}
    Find $E{y^2(t)}, R_{xy}(t_1,t_2), R_{yy}(t_1,t_2)$ if\\
    \begin{itemize}
        \item (a) (1) holds for all $t$
        \item (b) $y(0) = 0$ and (1) holds for $t \geq 0$
    \end{itemize}
\end{frame}

\section{Solution}
\begin{frame}{Solution}
    \begin{align}
        y(t) &= x(t) \times h(t)\\
        h(t) &= e^{-2t}U(t)
    \end{align}
\end{frame}

\begin{frame}{Part (a)}
    \begin{align}
        \implies E\{y^2(t)\} &= 5\times e^{-4t} U(t)\\
        E\{y^2(t)\} &= \frac{5}{4}\\
        \implies R_{xy}(t_1,t_2) &= 5\delta(t_1 - t_2)\times e^{-2t_2} U(t_2)\\
        R_{xy}(t_1,t_2) &= 5e^{-2(t_2 - t_1)} U(t_2 - t_1)\\
        R_{xy}(\tau) &= 5e^{-2\tau}U(\tau)\\
        \implies R_{yy}(t_1,t_2) &= 5e^{-2(t_2 - t_1)} U(t_2 - t_1) \times e^{-2t_1}U(t_1)\\
        R_{yy}(t_1,t_2) &= \frac{5}{4}e^{-2|t_2 - t_1|}\\
        R_{yy}(\tau) &= \frac{5}{4}e^{-2|\tau|}
    \end{align}
\end{frame}

\begin{frame}{Part (b)}
    For $t_1 < 0$ or $t_2 < 0$,
    \begin{align}
        R_{xy}(t_1,t_2) = 0\\
        R_{yy}(t_1,t_2) = 0
    \end{align}
\end{frame}
\begin{frame}{Part (b) - For $0 < t_1 < t_2$}
    \begin{align}
    \implies R_{xy}(t_1,t_2) &= 5\delta(t_1 - t_2)\times e^{-2t_2}\\
    R_{xy}(t_1,t_2) &= 5e^{-2t_2}\\
    \implies R_{yy}(t_1,t_2) &= \int_0^{t_1}5e^{-2(t_1 - \tau)}e^{-2(t_1 - \tau)}d\tau\\
    R_{yy}(t_1,t_2) &= \frac{5}{4}e^{-2(t_2 - t_1)}(1 - e^{-4t_1})
    \end{align}
\end{frame}
\begin{frame}{Part (b) - For $0 < t_2 < t_1$}
    \begin{align}
    \implies R_{xy}(t_1,t_2) &= 5\delta(t_1 - t_2)\times e^{-2t_1}\\
    R_{xy}(t_1,t_2) &= 5e^{-2t_1}\\
    \implies R_{yy}(t_1,t_2) &= \int_0^{t_1}5e^{-2(t_1 - \tau)}e^{-2(t_1 - \tau)}d\tau\\
    R_{yy}(t_1,t_2) &= \frac{5}{4}e^{-2(t_2 - t_1)}(1 - e^{-4t_1})
    \end{align}
\end{frame}
\end{document}